\documentclass[8pt,handout]{beamer}
\hyphenation{tro-pha-lla-ctique}


\usepackage[utf8x]{inputenc}
 \usepackage[T1]{fontenc}
\usepackage{wrapfig}
\usepackage{default}
% \usetheme{Goettingen}
\usepackage{amsmath}
\usecolortheme{rose} 
% \setbeamersize[width=0cm]{sidebar}


\usepackage{enumerate}
\usepackage[]{natbib}
\usepackage{graphicx}
\usepackage{wrapfig} 
\bibpunct{(}{)}{,}{a}{,}{,}
\usepackage{lmodern}
% \usepackage[colorlinks=true,urlcolor=blue,citecolor=green,linkcolor=blue,bookmarks=true]{hyperref}
\usepackage[french]{babel}
\author[]{Simon Carrignon}
\institute[]{
École~Pratique~des~Hautes~Études, Laboratoire de Cognition~Humaine~et~ARTificielle

\pgfdeclareimage[height=0.5cm]{ephe}{logo_ephe_large.jpg} %declare logo image with an alias here 
\pgfuseimage{ephe} \hfill \pgfdeclareimage[height=0.5cm]{chart}{logo-120px.png} %declare logo image with an alias here 
\pgfuseimage{chart}}
\usepackage[small]{caption}

\setbeamertemplate{caption}[numbered]
\usepackage{subfigure}
\usepackage{tikz}
\usetikzlibrary{decorations.pathreplacing}
\title[]{Proposition de solutions bas\'{e}es sur des heuristiques d'\'{e}change \'{e}n\'{e}rgetique pour la robotique autonome }
% \logo{\includegraphics[height=0.5cm]{logo_ephe_large.jpg}}
%\setbeamertemplate{sidebar right}[width = 0]%
%[top=structure.fg!05,bottom=structure.fg!90]

% \usepackage[footheight=1em]{beamerthemeboxes}
\setbeamersize{sidebar right width=0cm}
% \addfootboxtemplate{\color{black}}{\color{white}
%      \hfill\insertframenumber/11\hspace{2em}\null}
\begin{document}






\begin{frame}{}
\small

\begin{alertblock}{Question}
 
 Quelle modalité de comportement (échange vs non échange d'énergie) est la plus efficace pour maintenir le plus de robots ``en fonctionnement'' suivant les conditions de l'environnement? 
\end{alertblock}
\begin{columns}
\column{0.333\textwidth}


\column{0.333\textwidth}

\column{0.333\textwidth}
\end{columns}

\begin{columns}[t] 
\centering
\column{0.40\textwidth}
\begin{block}{Expériences}
\begin{enumerate}[ Exp I -]
  \item Quantité d'énergie par source (NF) faible~(5), 
  \item NF moyenne~(11), 
  \item NF abondante~(17). %=>  27 environnements
\end{enumerate}
\end{block}
% \vspace{2pt}
\begin{block}{Variable Dépendante } Nombre d'agents survivants à la fin de la simulation.\end{block}

\column{0.70\textwidth}


\begin{block}{Paramètres constants}
\begin{itemize}
 \item Nb d'agents initial : 100,
 \item Durée de la simulation : 1500 unités temporelles,
 \item Énergie initiale des agents : ${1\over{2}} \times{\mbox{Energie}_{max}}=100$,
 \item Nombre d'arbres à l'initialisation : 3.
%   \item Nombre d'obstacles (NO) : 20,

%   \item quantité d'énergie par source (NF) : faible~(5), moyenne~(11), abondante~(17,) %=>  27 environnements
\end{itemize}
\end{block}



\begin{block}{Paramètres variables (VI)}
\begin{enumerate}
  \item Modalité de comportement de la population initiale : 
    \begin{enumerate}[a -]%\scriptsize
    \item uniquement kind $\Rightarrow $ l'échange d'énergie est possible,
    \item uniquement selfish $\Rightarrow $ l'échange d'énergie est impossible.			%	=>  100 Simulation par type de population
%     \item 1/3 kind, 1/3 selfish, 1/3 TFT (population “mixte”).
   \end{enumerate}
  \item Vitesse de réapparition des sources (V) : 
  \begin{enumerate}[a -]
    \item lente~(3), 
    \item moyenne~(4), 
    \item rapide~(5).

  \end{enumerate}
\end{enumerate}
\end{block}



\end{columns}
\centering

$ \underbrace{ \mbox{100 simulations\;}\; \times \; \mbox{ 2 Populations } \; \times \; \mbox{ 3 Expériences }  \; \times \; \mbox{9 environnements}}_{\mbox{5400 Simulations}} $
	


\end{frame}



\end{document}
 